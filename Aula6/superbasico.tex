\documentclass{article}
\usepackage[latin1,utf8]{inputenc}
\begin{document}
%
\begin{flushleft}
{\bf Métodos Computacionais em Física I - 2009/1}\\
\end{flushleft}
\begin{flushright}
{\bf nome:} Sandra Amato\\
\end{flushright}
\begin{center}
\Huge{\bf O Maior}\\
\huge{\bf Enorme}\\
\Large{\bf Muito grande}\\
\large{\bf grande}\\
\small{\bf pequeno}\\
\footnotesize{\bf pé de página}\\ 
\scriptsize{\bf o menor}
\end{center}
Este texto é um exemplo de utilização muito básico mesmo. Para escrever
 u\-sa\-mos %separação de sílabas 
o modo parágrafo, e para colocar fórmulas e diversos símbolos 
 matemáticos temos que entrar no modo matemático, o que pode
ser 
feito de muitas maneiras.

Veja como obter o recuo de parágrafo. E agora como mudar de linha\\ simplesmente.

Agora um exemplo simples do modo matemático:
$$ \int_0^1x dx=\frac{1}{2}. $$
Repare como a a fonte do modo matemático é diferente (x ou $x$), e como 
o espaço foi ignorado. Letras gregas podem ser escritas em modo matemático ($\alpha$, $\beta$, ...).
\end{document}

