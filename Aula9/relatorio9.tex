\documentclass[12pt,a4paper]{article}
\usepackage[latin1]{inputenc}
\usepackage[portuguese]{babel}
\usepackage{fullpage}
\usepackage[affil-it]{authblk}
\usepackage{graphicx}
\usepackage{placeins}

\title{\textbf{Velocidade de um corpo sob a��o de for�a viscosa usando o m�todo de Euler}}
\author{Natasha M. da Rocha}
\affil{Universidade Federal do Rio de Janeiro}
\date{\today}

\begin{document}

\maketitle

\section{Movimento de um corpo em meio viscoso}
Uma bolha em meio viscoso (xampu, por exemplo) sofre a��o de tr�s for�as: da gravidade, empuxo e uma for�a de atrito devido � viscosidade. Assim, pela segunda lei de Newton:
\begin{equation}
\rho_{ar} V\frac{d\vec{v}}{dt} = \rho_x Vg - \rho_{ar} Vg - bv^r
\end{equation}
Considerando o volume da bolha igual a $\frac{4}{3}\pi R^3$:
\begin{equation}
\frac{d\vec{v}}{dt} = \frac{\rho_x-\rho_{ar}}{\rho_{ar}} g - \frac{b}{\frac{4}{3}\pi\rho_{ar}R^3}v^r
\end{equation}
Onde $\rho_x$ � a densidade do meio viscoso, $\rho_{ar}$ � a densidade do ar, 

\section{M�todo de Euler}
Pela dificuldade de se calcular analiticamente a Equa��o Diferencial Ordin�ria (EDO) [2], usamos o M�todo de Euler para calcul�-la numericamente. O m�todo consiste em aproximar a fun��o em uma reta em um ponto dado e estimar um pr�ximo ponto a partir da mesma, repetindo o processo para o pr�ximo ponto, conforme indicado na figura abaixo:
\FloatBarrier
\begin{figure}[!ht]
\center
\includegraphics[scale=0.18]{euler.png}
\end{figure}
\FloatBarrier
A dist�ncia entre os pontos define a precis�o, ou seja, quanto menor a mesma, mais precisa � a aproxima��o - em contrapartida, maior � o n�mero de pontos a serem encontrados -. A f�rmula para a utiliza��o do m�todo � a seguinte:
\begin{equation}
y_{i+1} = y_i + hf(x_i,y_i)
\end{equation}
Onde $f(x_i,y_i)$ � o valor da derivada de $y$ em fun��o de $x$.

\section{Resultados}
Consideramos $\rho_{ar} = 1.25*10^{-6} gmm^{-3}$, $\rho_x=1.03*10^{-3} gmm^{-3}$, $g=9.78*10^{-3}mms^{-2}$, $r = 1.7$ e $b = 420gs^{-0.3}mm^{-0.7}$. Para uma bolha com di�metro de $5mm$ e usando o M�todo de Euler, encontramos os valores contidos no seguinte grafico:

\FloatBarrier
\begin{figure}[!ht]
\center
\includegraphics{bolhas.pdf}
\end{figure}
\FloatBarrier

\section{Conclus�es}
Usando um passo de $10^-7$ s, encontramos que a velocidade limite para o qual a fun��o converge � igual � $1.302833$ mm/s. Descobrimos tamb�m que usando um passo de $2*10^{-8}$s, a velocidade converge nessa velocidade limite.

\end{document}